\documentclass{sig-alternate-05-2015}


\begin{document}

\setcopyright{acmcopyright}

\conferenceinfo{FSE '16}{November 13--19, 2016, Seattle, WA, USA}


\title{A Doomsday Vault of Software Engineering Tools}
\subtitle{Archiving Software Engineering Tools from ICSE and FSE 2011 through 2014}

\numberofauthors{75}

\author{
\alignauthor
Emerson Murphy-Hill\\
       \affaddr{Department of Computer Science}\\
       \affaddr{NC State University}\\
       \affaddr{Raleigh, NC, USA 27695}\\
       \email{emerson@csc.ncsu.edu}}

\additionalauthors{Additional authors: 
%Example:
% John Smith (The Th{\o}rv{\"a}ld Group,
% email: {\texttt{jsmith@affiliation.org}}) and Julius P.~Kumquat
% (The Kumquat Consortium, email: {\texttt{jpkumquat@consortium.net}}).
%Alphabetical:
Shabbir Abdul (\url{sabdul@ncsu.edu}), 
Varun Aettapu (\url{vaettap@ncsu.edu}), 
Sumeet Agarwal (\url{sagarwa6@ncsu.edu}), 
Sindhu Anangur Vairavel (\url{sanangu@ncsu.edu}), 
Rishi Avinash Anne (\url{raanne@ncsu.edu}), 
Haris Mahmood Ansari (\url{hmansari@ncsu.edu}), 
Ankit Bhandari (\url{abhanda3@ncsu.edu}), 
Anand Bhanu (\url{bhanua@ncsu.edu}), 
Aditya Vinayak Bhise (\url{avbhise@ncsu.edu}), 
Saikrishna Teja Bobba (\url{sbobba3@ncsu.edu}), 
Vineela Boddula (\url{vboddul@ncsu.edu}), 
Venkata Krishna Sailesh Bommisetti (\url{vbommis@ncsu.edu}), 
Dwayne Christian (Chris) Brown (\url{dcbrow10@ncsu.edu}), 
Peter Morgan Chen (\url{pmchen@ncsu.edu}), 
Yi Chun (\url{Yi-Chun}) Chen (\url{ychen74@ncsu.edu}), 
Nikhil Chinthapallee (\url{nchinth@ncsu.edu}), 
Karan Singh Dagar (\url{kdagar@ncsu.edu}), 
Joseph Decker (\url{jdecker@ncsu.edu}), 
Pankti Rakeshkumar Desai (\url{prdesai2@ncsu.edu}), 
Jayant Dhawan (\url{jdhawan2@ncsu.edu}), 
Yihuan Dong (\url{ydong2@ncsu.edu}),
Sarah Elizabeth Elder (\url{seelder@ncsu.edu}), 
Shrenuj Gunvant Gandhi (\url{sgandhi4@ncsu.edu}), 
Jennifer Michelle Green (\url{jmgree17@ncsu.edu}), 
Mohammed Hasibul Hassan (\url{mhhassan@ncsu.edu}), 
Satish Inampudi (\url{sinampu@ncsu.edu}), 
Pragyan Paramita Jena (\url{ppjena@ncsu.edu}), 
Bhargav Rahul (\url{Bhargav}) Jhaveri (\url{bjhaver@ncsu.edu}), 
Apoorv Vijay Joshi (\url{avjoshi@ncsu.edu}), 
Nikhil Josyabhatla (\url{njosyab@ncsu.edu}), 
Sujith Katakam (\url{skataka@ncsu.edu}), 
Juzer Husainy Khambaty (\url{jhkhamba@ncsu.edu}), 
Aneesh Arvind Kher (\url{aakher@ncsu.edu}), 
Craig Kimpel (\url{ckimpal@ncsu.edu}), 
Siddhartha Kollipara (\url{skollip@ncsu.edu}), 
Asish Prabhakar Kottala (\url{akottal@ncsu.edu}), 
Abishek Kumar (\url{akumar21@ncsu.edu}), 
Harini Reddy Kumbum (\url{hkumbum@ncsu.edu}), 
Nitish Pradeep Limaye (\url{nplimaye@ncsu.edu}), 
Apoorv Mahajan (\url{amahaja3@ncsu.edu}), 
Sai Sindhur Malleni (\url{smallen3@ncsu.edu}), 
Sudha Manchukonda (\url{smanchu@ncsu.edu}), 
Kavit Maral Mehta (\url{kmmehta@ncsu.edu}), 
Justin Alan Middleton (\url{jamiddl2@ncsu.edu}), 
Ramakant Moka (\url{rmoka@ncsu.edu}), 
Eesha Gopalakrishna Mulky (\url{egmulky@ncsu.edu}), 
Gauri Naik (\url{gnaik2@ncsu.edu}), 
Shraddha Anil Naik (\url{sanaik2@ncsu.edu}), 
Yashwanth Nallabothu (\url{ynallab@ncsu.edu}), 
Yogesh Nandakumar (\url{ynandak@ncsu.edu}), 
Kairav Sai Padarthy (\url{kspadart@ncsu.edu}), 
Pulkesh Kumar Yadav Pannalal (\url{ppannal@ncsu.edu}), 
Sattwik Pati (\url{spati2@ncsu.edu}), 
Kahan Prabhu (\url{kprabhu@ncsu.edu}), 
Shashank Goud Pulimamidi (\url{spulima@ncsu.edu}), 
Gargi Sandeep Rajadhyaksha (\url{gsrajadh@ncsu.edu}), 
Priyadarshini Rajagopal (\url{prajago4@ncsu.edu}), 
Venkatesh Sambandamoorthy (\url{vsamban@ncsu.edu}), 
Mohan Sammeta (\url{msammet@ncsu.edu}), 
Shaown Sarker (\url{ssarker@ncsu.edu}), 
Anshita Sayal (\url{asayal@ncsu.edu}), 
Vrushti Kamleshkumar Shah (\url{vkshah@ncsu.edu}), 
Esha Sharma (\url{esharma2@ncsu.edu}), 
Saurav Shekhar (\url{sshekha3@ncsu.edu}), 
Sarthak Prabhakar Shetty (\url{spshetty@ncsu.edu}), 
Manish Ramashankar Singh (\url{mrsingh@ncsu.edu}), 
Ankush Kumar Singh (\url{asingh21@ncsu.edu}), 
Vinay Kumar Suryadevara (\url{vksuryad@ncsu.edu}), 
Sumit Kumar Tomer (\url{sktomer@ncsu.edu}), 
Akriti Tripathi (\url{atripat4@ncsu.edu}), 
Jennifer Tsan (\url{jtsan@ncsu.edu}), 
Vivekananda Vakkalanka (\url{vvakkal@ncsu.edu}), 
Alexander Valkovsky (\url{avalkov@ncsu.edu}), 
Rishi Kumar Vardhineni (\url{rkvardhi@ncsu.edu}) and
Manav Verma (\url{mverma4@ncsu.edu}).}
%\date{30 July 1999}

\maketitle
\begin{abstract}
Many innovative software engineering tools appear at the field's premier venues, the 
International Software Engineering Conference (ICSE) and the 
Foundations of Software Engineering (FSE).
But what happens to these tools after they were presented?
In this paper, we spend 10,000 hours %update with real number 
trying to obtain, download, use, and repackage 150 %update
tools from ICSE and FSE's tool demonstration tracks.
Our results enumerate the practical and accidental reasons that
software engineering tools fail to work over time,
and provide practical implications for creating lasting tools.
\end{abstract}


%
% The code below should be generated by the tool at
% http://dl.acm.org/ccs.cfm
\begin{CCSXML}
<ccs2012>
<concept>
<concept_id>10002944.10011123.10010912</concept_id>
<concept_desc>General and reference~Empirical studies</concept_desc>
<concept_significance>300</concept_significance>
</concept>
</ccs2012>
\end{CCSXML}

\ccsdesc[300]{General and reference~Empirical studies}

\printccsdesc

\keywords{Software engineering tools; replication}

\section{Introduction}

Sofware engineering research seeks to better
understand software and how it is built and constructed.
While such understanding in isolation can be insightful,
ultimately a substantial amount of such research 
aims to impact the practice of software engineering.
To do so, researchers can create recommendations
for new software engineering practices, can 
create educational techniques and materials, 
and can create tools.

Arguably creating new tools is the most common
way that sofware engineering researchers
attempt to influence practice.
% evidence? would be good to see what the primary contributions are of a recent ICSE or FSE?
Broadly speaking, tools are software that can
help design and build software.
Examples include a tool that helps mobile application
developers choose which devices to target~\cite{prada},
a tool that checks the use of locks in multithreaded programs~\cite{ernst},
and a tool that creates code from natural language text~\cite{desai}.

While papers describe tools in software engineering venues,
there are several reasons why software engineering researchers 
should make the tools themselves available.
First, the full details of how the tool works, both in terms of its internals
and from an end-user perspective, may not be clear from the paper.
Second, making the tools available can help pracitioners try and adopt
the tools in their work.
Third, it helps facilitate reproducibility by enabling future researchers
to perform studies with the original tools.
Finally, it helps the advancement of the field by allowing others to build on 
existing tools, rather than re-buiding them from scratch.

Despite the benefits of making tools available, in this paper
we catalog the practical difficulties in doing so.
We describe a study that examined the tools presented at the 
premier venues for software engineering research over the 
past several years.
As part of a graduate university course on software engineering, 
we spent 10,000 hours trying to obtain, download, use
and repackage tools from the International Conference
on Sofware Engineering (ICSE) and the Symposium on the Foundations
of Software Engineering (FSE).
In doing so, we make the following three main 
contributions in this paper:

\begin{itemize}
  \item A study that evaluates the difficulty in getting tools 
  		working across a variety of software engineering research.
  \item A synergistic course project for graduate software engineering students 
		that gives students meaningful educational value \emph{and}
		provides the research community value.
  \item XXX existing tools repackaged in automatically-built virtual machines
  		to make it easier for others to use these tools.
\end{itemize}

\section{Related Work}

timeliness

General scientific interest in reproducibility.
How it fits with other Rs.
State of practice.

Software reproducibility outside SE.
Systems~\cite{collberg2016repeatability,proebsting2015repeatability}

Others~\cite{kovacevic2007encourage,vandewalle2009reproducible,stodden2009enabling,klein2012run}.


Arguably as SE researchers we should be the best!

Other fields:
- Data science (doing better - http://zenodo.org)
- MPC journal requires code submission (tarball)
- In general, Math not doing a lot with code sub submissions
- Security, should be able to do VM, but don't (but see Will's paper)
- HPC doesn't (and maybe can't)
- RTC is starting to do it, but maybe shouldn't

`` We argue that, with some exceptions, anything less than the release of source programs is intolerable for results that depend on computation.''~\cite{ince2012case}.

Constant worry is protecting IP in security and HPC

Repeatability in AI-SE: ``depends on who does it''
Reproducibility assessment for 2 papers, plus reproducibility overview~\cite{gonzalez2012reproducibility}


While no studies of software reproducibility in our field,
practical interest.

Artifact evaluation committees. (SIGPLAN, where else?)


\section{Research Questions}

Henceforth, we will simply say \textit{tool}
to refer to software engineering tools presented at
the International Conference on Software Engineering
or Foundations of Software Engineering in their
respective tool demonstration tracks.

\begin{enumerate}
  \item How much effort is required to get tools to work?
  \item What are the barriers to get tools to work?
  \item How much effort is required to get tools to work in virtual machines?
  \item What are the barriers to get tools to work in virtual machines?
\end{enumerate}

\section{Procedure}

No cost, but demos ok. (How many did this happen with?)

Tools where tool could convievably be put into a virtual machine. How many didn't fit?

\section{Results}

\subsection{RQ1 and RQ2}

\subsubsection{Effort}

Average person-hours per tool, distribution (min, max, box plot)

Time invested in evaluating non-working tools.

\subsubsection{Challenge: Tool Cannot Be Obtained}

What percent of tool links were dead?
What percent of tools said they were available in the paper, but the tool could not be obtained?
What percent of tools were being planned to be commercialized? What were actually commercial?
What percent of tools could we use if we had paid for them?

\subsubsection{Challenge: Disappearing Tools}

One author said tool just doesn't exist anymore.

Another author had to dig tool out of long term archive.

Several authors (what percent?) has tools hosted on Google code, even though it was dying.
Did we save 'em?

\subsubsection{Challenge: Author non-responsive}

%TODO: look at righthand column for tool; indications of why tools not working?

\subsubsection{Challenge: Technical Difficulties}

\subsubsection{Challenge: Inconsistencies}

One tool required to different VMs because two features needed different prereqs.

Some tools versioned differently.

\subsection{RQ3 and RQ4}


\subsubsection{Challenge: Tool Licensing}

University grey area.

\subsubsection{Challenge: Technology Stack Licensing}

\subsubsection{Challenge: Author Doesn't Want Redistribution}

Even when tool is available

\section{Conclusions}

Some other things.

%ACKNOWLEDGMENTS are optional
%\section{Acknowledgments}


\bibliographystyle{abbrv}
\bibliography{references}

\end{document}
